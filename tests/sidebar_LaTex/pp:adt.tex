\subsection*{ Abstract Datatypes }

\subsubsection*{ Intro }

An Abstract Datatype relies on \textit{functions} to describe the possible values of a type. We start with a simplistic example:


\begin{tcblisting}{ arc=0pt, outer arc=0pt, listing only, breakable}
data Nat = Zero | Succ Nat

\end{tcblisting}

\begin{itemize}
	\item  the expression \texttt{data Nat} introduces a new \textbf{type} in the programming language
	\item  after \texttt{=}, the \textbf{base constructors} of the type follow. In Haskell, \textbf{all constructors} must begin with a capital letter
	\item  \texttt{Zero} is a nullary constructor. 
	\item  \texttt{Succ Nat} designates an \textbf{internal} constructor, which expects a natural number (of type \texttt{Nat}). A value \texttt{(Succ x)} \textbf{is} of type \texttt{Nat} (i.e. a natural number), and we may be tempted to see it as a function call, which \textit{returns} the successor of \texttt{x}
\end{itemize}

\texttt{Zero} and \texttt{Succ} are called \textbf{data constructors} in Haskell. \texttt{Nat} is called a \texttt{type} or \texttt{type-constructor}. We shall distinguish between the two in a later lecture.

Note that the \textit{internal} representation of an ADT, as perceived by the programmer, is \textit{abstract}. We may see values as \textit{calls} of special functions - data constructors. Except for their meaning (and language-level implementation), data constructors behave exactly as functions. For instance:
\begin{itemize}
	\item  \texttt{Zero :: Nat}
	\item  \texttt{Succ :: Nat -\textgreater  Nat}
\end{itemize}

We continue the example with addition:


\begin{tcblisting}{ arc=0pt, outer arc=0pt, listing only, breakable}
add :: Nat -> Nat -> Nat
add Zero y = y
add (Succ x) y = Succ (add x y)

\end{tcblisting}


An important observation is that the \textbf{pattern matching mechanism} in Haskell relies on data constructors, and their applications. For instance, the following definition is a correct usage of the pattern matching mechanism:

\begin{tcblisting}{ arc=0pt, outer arc=0pt, listing only, breakable}
f (1:y:[]) = ...

\end{tcblisting}

I uses the data constructors \texttt{(:)} and \texttt{[]} for lists, as well as the data constructor \texttt{1} for integers. The pattern describes any list of integers which starts with a \texttt{1} and contains exactly two elements.

\subsubsection*{ Monomorphic List implementation }

In what follows, we give an implementation for the type \textit{List of integers}. This type is called \textit{monomorphic}, since our list can only contain elements of a single type (integer):


\begin{tcblisting}{ arc=0pt, outer arc=0pt, listing only, breakable}
data IList = Void | Cons Integer IList

app :: IList -> IList -> IList
app Void l = l
app (Cons h t) l = Cons h (app t l)

convert :: IList -> [Integer]
convert Void = []
convert (Cons h t) = h : (convert t)

mfoldl :: (b -> Integer -> b) -> b -> IList -> b
mfoldl op acc Void = acc
mfoldl op acc (Cons h t) = mfoldl op (op acc h) t

mfoldr :: (Integer -> a -> a) -> a -> IList -> a
mfoldr op acc Void = acc
mfoldr op acc (Cons h t) = op h (mfoldr op acc t)

convert2 :: IList -> [Integer]
convert2 = mfoldr (:) []

showl :: IList -> String
showl = show . convert2


\end{tcblisting}


In the above code, we have defined some basic list operations, such as \texttt{app} (list concatenation) and \texttt{convert}, which transforms a list of type \texttt{IList} to a conventional Haskell list (of type \texttt{[Integer]}).

We have also implemented the two folding procedures for \texttt{IList}. Note the type signature of each fold. Finally, we have used folds to provide an alternative implementation for convert, as well as for converting lists to strings.


\subsubsection*{ Propositional Logic in Haskell }

Abstract Datatypes are a natural way to define more elaborate data-structures, such as propositional formulae. We give a possible definition below:


\begin{tcblisting}{ arc=0pt, outer arc=0pt, listing only, breakable}
data Formula = Var String | And Formula Formula | Or Formula Formula | Not Formula

\end{tcblisting}


We observe that:
\begin{itemize}
	\item  \texttt{Var :: String -\textgreater  Formula}
	\item  \texttt{And :: Formula -\textgreater  Formula -\textgreater  Formula}
	\item  \texttt{Or :: Formula -\textgreater  Formula -\textgreater  Formula}
	\item  \texttt{Not :: Formula -\textgreater  Formula}
\end{itemize}

A good exercise consists in the implementation of a display function for formulae:

\begin{tcblisting}{ arc=0pt, outer arc=0pt, listing only, breakable}
fshow :: Formula -> String
fshow (Var v) = v
fshow (And f1 f2) = "("++(fshow f1)++" ^ "++(fshow f2)++")"
fshow (Or f1 f2) = "("++(fshow f1)++" V "++(fshow f2)++")"
fshow (Not f) = "~("++(fshow f)++")"

\end{tcblisting}


Next, we implement a function \texttt{push}, which pushes negation \textit{inward}:

\begin{tcblisting}{ arc=0pt, outer arc=0pt, listing only, breakable}
push :: Formula -> Formula
push (Var v) = (Var v)
push (Not (Var v)) = Not (Var v)
push (Not (And f1 f2)) = Or (push (Not f1)) (push (Not f2))
push (Not (Or f1 f2)) = And (push (Not f1)) (push (Not f2))
push (And f1 f2) = And (push f1) (push f2)
push (Or f1 f2) = Or (push f1) (push f2)

\end{tcblisting}


Notice, at lines 4,5 the implementation of deMorgan's laws.

Finally, we implement a function which computes the truth-value of a formula, under a certain interpretation. First, we define the type \texttt{Interpretation}:

\begin{tcblisting}{ arc=0pt, outer arc=0pt, listing only, breakable}
type Interpretation = String -> Bool

i :: Interpretation
i "x" = True
i "y" = False
i "z" = True

\end{tcblisting}
 

In the first line, we have defined a \textbf{type-alias}: \texttt{Interpretation} is a function from strings to booleans. We have also implemented a three-variable interpretation, for testing purposes. Next, we implement \texttt{eval}:


\begin{tcblisting}{ arc=0pt, outer arc=0pt, listing only, breakable}
eval :: Interpretation -> Formula -> Bool
eval i (Var v) = i v
eval i (Not f) = not(eval i f)
eval i (And f1 f2) = (eval i f1) &\&&&\&& (eval i f2)
eval i (Or f1 f2) = (eval i f1) || (eval i f2)

\end{tcblisting}


\subsubsection*{ Monomorphic Trees in Haskell }


\begin{tcblisting}{ arc=0pt, outer arc=0pt, listing only, breakable}
data ITree = Leaf | Node ITree Integer ITree

\end{tcblisting}


Note that:
\begin{itemize}
	\item  \texttt{Leaf :: ITree}
	\item  \texttt{Node :: ITree -\textgreater  Integer -\textgreater  ITree -\textgreater  ITree}
\end{itemize}

Next, we implement a folding operation on Trees. The key to it is to conceptually define what folding should do on Trees. To grasp an intuition, recall that: 
\begin{itemize}
	\item  \texttt{foldr (:) []} is the \textbf{identity function} on lists. Hence, \texttt{foldr (:) [] [1,2,3]} produces \texttt{[1,2,3]}.
	\item  also, recall that the map operation can be defined as: \texttt{\textbackslash f -\textgreater  foldr ((:).f) []}
\end{itemize}

Similar to the list case, a fold on trees should:
\begin{itemize}
	\item  \textbf{preserve the tree structure}, given the appropriate operator (e.g. \texttt{Node}). 
	\item  hence, the call \texttt{mtfold Node Leaf} (where \texttt{Leaf} is the accumulator) should be the \textbf{identity function on trees}. 
\end{itemize}

Let us define the identity function \texttt{tid} for trees:

\begin{tcblisting}{ arc=0pt, outer arc=0pt, listing only, breakable}
tid Leaf = Leaf
tid (Node r k l) = Node (tid r) k (tid l)

\end{tcblisting}


As already said, \texttt{(tid t)} is equivalent to the call \texttt{(mtfold Node Leaf t)}, for any arbitrary tree \texttt{t}.
To obtain the general \texttt{mtfold} implementation, we simply generalise \texttt{Node} by an arbitrary operation \texttt{op}:
\begin{itemize}
	\item  if \texttt{Node :: ITree -\textgreater  Integer -\textgreater  ITree -\textgreater  ITree}, then
	\item  \texttt{op :: b -\textgreater  Integer -\textgreater  b -\textgreater  b}
\end{itemize}

We also generalise \texttt{Leaf} by an arbitrary accumulator:
\begin{itemize}
	\item  if \texttt{Leaf :: ITree} then
	\item  \texttt{acc :: b}
\end{itemize}

The code generalisation becomes:


\begin{tcblisting}{ arc=0pt, outer arc=0pt, listing only, breakable}
mtfold :: (b -> Integer -> b -> b) -> b -> ITree -> b
mtfold op acc = 
    let f (Node left key right) = f (f left) key (f right)
        f Leaf = acc
    in f

\end{tcblisting}


Notice that \texttt{f} is precisely the generalisation of \texttt{tid} according to our observations. We can use \texttt{mtfold} to implement various tree operations such as:


\begin{tcblisting}{ arc=0pt, outer arc=0pt, listing only, breakable}
tsum = mtfold (\k r l->k + r + l) 0 
tmirror = mtfold (\k r l -> Node l k r) Leaf 
tflatten = mtfold (\k r l -> r ++ [k] ++ l) [] 

\end{tcblisting}
 