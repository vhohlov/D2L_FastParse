\section*{ Funcții }

Scopul laboratorului:
\begin{itemize}
	\item  Semnificația termenului \textit{aplicație}
	\item  Definirea funcțiilor anonime în Haskell
	\item  Curry vs uncurry (și de ce nu contează în Haskell)
	\item  Combinatori vs Închideri funcționale
\end{itemize}
\subsection*{ Funcții de ordin superior }

Funcțiile de ordin superior sunt funcții care lucrează cu alte funcții: le primesc ca parametrii sau le returnează.

Pentru a înțelege importanța lor, vom da următorul exemplu: ne propunem să scriem două funcții care primesc o listă și returnează, într-o nouă listă
\begin{itemize}
	\item  toate elementele pare
	\item  toate elementele mai mari decât 10
\end{itemize}


\begin{tcblisting}{ arc=0pt, outer arc=0pt, listing only, breakable}
evenElements [] = []
evenElements (x:xs) = if even x
                      then x : evenElements xs
                      else evenElements xs
                      
greaterThan10 [] = []
greaterThan10 (x:xs) = if x > 10
                       then x : greaterThan10 xs
                       else greaterThan10 xs

\end{tcblisting}


\begin{tcolorbox}[colback=blue!10, colframe=blue!20]
În loc de \texttt{if}, puteți folosi următoarea sintaxă:

\begin{tcblisting}{ arc=0pt, outer arc=0pt, listing only, breakable}
myFunction x
        | x < 10    = "One digit"
        | x < 100   = "Two digits"
        | x < 1000  = "Three digits"
        | otherwise = "More than four digits"

\end{tcblisting}

\end{tcolorbox}

Testăm funcțiile scrise:

\begin{tcblisting}{ arc=0pt, outer arc=0pt, listing only, breakable}
*Main> evenElements [1..20]
[2,4,6,8,10,12,14,16,18,20]
*Main> greaterThan10 [1..20]
[10,11,12,13,14,15,16,17,18,19,20]

\end{tcblisting}


Observăm că funcțiile definite mai sus sunt foarte asemănătoare. De fapt, doar condiția verificată în \texttt{if} diferă. Scriem, deci, o funcție generală care primește o funcție pentru testarea elementelor:


\begin{tcblisting}{ arc=0pt, outer arc=0pt, listing only, breakable}
-- In primul pattern, nu folosim functia de testare, deci nu ne intereseaza ca aceasta
-- sa fie legata la un nume, lucru marcat prin "_"
myFilter _ [] = []
myFilter test (x:xs) = if test x
                       then x : myFilter test xs
                       else myFilter test xs

\end{tcblisting}


Acum putem rescrie funcțiile noastre, într-un mod mai elegant, utilizând funcția de filtrare:


\begin{tcblisting}{ arc=0pt, outer arc=0pt, listing only, breakable}
evenElements = myFilter even

greaterThan10 = myFilter (> 10)

\end{tcblisting}



\subsection*{ Currying vs. uncurrying }

Currying (numit după \texttt{tizul Haskell-ului} \footnote{\url{https://en.wikipedia.org/wiki/Haskell\_Curry}}) este procesul prin care, dintr-o funcție care ia mai multe argumente, se obține o secvență de funcții care iau un singur argument.

De exemplu, dintr-o funcție de două argumente \texttt{f : X × Y → Z} se obține o funcție\\\texttt{curry(f) : X → (Y → Z)}. Noua funcție \texttt{curry(f)} primește un argument de tipul \texttt{X} și întoarce o funcție de tipul \texttt{Y → Z} (adică o funcție care primește un argument de tipul \texttt{Y} și întoarce un rezultat de tip \texttt{Z}). Considerând operatorul \texttt{→} asociativ la dreapta, putem omite parantezele, i.e. \texttt{curry(f) : X → Y → Z}.

Operația inversă se numește "uncurry": \texttt{f : X → Y → Z,   uncurry(f): X × Y → Z}.

Întorcându-ne la funcția de filtrare definită mai devreme, care este tipul ei?


\begin{tcblisting}{ arc=0pt, outer arc=0pt, listing only, breakable}
*Main> :t myFilter
myFilter :: (a -> Bool) -> [a] -> [a]

\end{tcblisting}


Și în \texttt{laboratorul trecut} \footnote{\url{http://www.cfvbfbtlotto.com/dokuwiki/doku.php?id=pp:l01}}, am observat că nu există o separare între domeniu și codomeniu, de genul\\\texttt{(a -\textgreater  Bool) x [a] -\textgreater  [a]}.

\begin{tcolorbox}[colback=yellow!40, colframe=yellow!60, breakable]
Acest lucru se datorează faptului că, în Haskell, \textbf{toate funcțiile iau un singur argument}. Alternativ, putem spune despre ele că sunt \textit{curried}.

Tipul funcției noastre trebuie interpretat astfel:\\primește ca argument o funcție de tipul \texttt{a -\textgreater  Bool} (ia un argument și întoarce o booleană) și întoarce o funcție de tipul \texttt{[a] -\textgreater  [a]} (ia o listă și întoarce o listă de același tip).

De aceea, în exemplul de mai sus am putut definit \texttt{evenElements} (o funcție care ia o listă și returnează o listă) ca fiind \texttt{myFilter even} care are exact acest tip, i.e. \texttt{[a] -\textgreater  [a]}.
\end{tcolorbox}

\begin{tcolorbox}[colback=blue!10, colframe=blue!20]
Haskell pune la dispoziție funcțiile \texttt{curry} și \texttt{uncurry}. 


\begin{tcblisting}{ arc=0pt, outer arc=0pt, listing only, breakable}
*Main> :t curry
curry :: ((a, b) -> c) -> a -> b -> c
*Main> :t uncurry
uncurry :: (a -> b -> c) -> (a, b) -> c

\end{tcblisting}


Funcțiile primite, respectiv returnate de \texttt{curry} și \texttt{uncurry} iau tot un singur argument, numai că acesta este un tuplu.


\begin{tcblisting}{ arc=0pt, outer arc=0pt, listing only, breakable}
*Main> let evenElements = (uncurry myFilter) even

<interactive>:12:39:
    Couldn't match expected type '(a0 -> Bool, [a0])'
                with actual type 'a1 -> Bool'
    In the second argument of 'uncurry', namely 'even'
    In the expression: (uncurry myFilter) even
    In an equation for 'evenElements':
        evenElements = (uncurry myFilter) even
*Main> let evenElements l = (uncurry myFilter) (even, l)
*Main>

\end{tcblisting}

\end{tcolorbox}

\subsection*{ Funcții anonime }

Ne propunem să scriem o funcție care primește o listă și întoarce toate elementele ei divizibile cu 5. Având deja o funcție de filtrare, putem scrie:


\begin{tcblisting}{ arc=0pt, outer arc=0pt, listing only, breakable}
testDiv5 x = mod x 5 == 0
multiplesOf5 = myFilter testDiv5

\end{tcblisting}


Această abordare funcționează, însă am poluat spațiul de nume cu o funcție pe care nu o folosim decât o singură dată - \texttt{testDiv5}.

Funcțiile anonime (cunoscute și ca expresii lambda) sunt funcții fără nume, folosite des în lucrul cu funcții de ordin superior.

În Haskell, se folosește sintaxa: \texttt{\textbackslash x -\textgreater  x + 5}. Funcția definită ia un parametru și returnează suma dintre acesta și 5. Caracterul \texttt{\textbackslash } este folosit pentru că seamănă cu λ (lambda).

Rescriind funcția noastră, obținem:

\begin{tcblisting}{ arc=0pt, outer arc=0pt, listing only, breakable}
multiplesOf5 = myFilter (\x -> mod x 5 == 0)

\end{tcblisting}


Următoarele expresii sunt echivalente:


\begin{tcblisting}{ arc=0pt, outer arc=0pt, listing only, breakable}
f x y = x + y
f x = \y -> x + y
f = \x y -> x + y

\end{tcblisting}


\begin{tcolorbox}[colback=blue!10, colframe=blue!20]
Nu există vreo diferență între \texttt{\textbackslash x y -\textgreater  x + y} și \texttt{\textbackslash x -\textgreater  \textbackslash y -\textgreater  x + y}.
\end{tcolorbox}

\subsection*{ Combinatori și închideri funcționale }

Un \textbf{combinator} este o funcție care nu se folosește de \textit{variabile libere}.  O variabilă este liberă în definiția unei funcții dacă nu se \textit{referă} la niciunul dintre parametrii formali ai acesteia.


\begin{tcblisting}{ arc=0pt, outer arc=0pt, listing only, breakable}
-- Exemple de combinatori
f1 = \a -> a 
         -- prin f1 &î&i d&ă&m un nume func&ț&iei anonime de dup&ă& egal
         -- aceast&ă& func&ț&ie anonim&ă& are un parametru formal, "a"
         -- defini&ț&ia ei con&ț&ine doar evaluarea lui a
         -- este un combinator deoarece nu folose&ș&te nicio alt&ă& variabil&ă& &î&n afar&ă& de a

f2 = \a -> \b -> a
f3 = \f -> \a -> \b -> f b a

-- Exemple de "ne-combinatori"
f1_no = \a -> a + x
  where x = some_other_expr
  -- x este o variabil&ă& liber&ă& &î&n contextul defini&ț&iei func&ț&iei anonime de dup&ă& egal

-- Pentru mai multe exemeple, continu&ă& s&ă& cite&ș&ti

\end{tcblisting}


Închiderile funcționale (\textbf{closures}) sunt opusul combinatorilor - se folosesc de variabile libere în definiția lor.  Cu alte cuvinte, sunt funcții care, pe lângă definiție, conțin și un \textbf{environment} de care se folosesc; acest environment conține variabilele libere despre care vorbeam.  Denumirea provine din faptul că environment-ul nu este menționat explicit, el este determinat implicit; spunem că funcția \textbf{se închide} peste variabilele (\textit{libere}) \textbf{a}, \textbf{b}, etc.  

Diferența dintre combinatori și închideri funcționale este de multe ori subtilă.  Să aruncăm o privire asupra unuia dintre exemplele de mai sus.


\begin{tcblisting}{ arc=0pt, outer arc=0pt, listing only, breakable}
flip = \f -> \a -> \b -> f b a -- was f3

\end{tcblisting}


După cum am spus, acesta este un combinator.  Ținând cont de faptul că în Haskell \textbf{currying}-ul funcțiilor (gândiți-vă la \textit{parantezarea tipului}) nu contează, putem rescrie această funcție astfel:


\begin{tcblisting}{ arc=0pt, outer arc=0pt, listing only, breakable}
flip f = \a -> \b -> f b a
      -- ^^^^^^^^^^^^+^^^^ 
      --     f este o variabil&ă& liber&ă& &î&n contextul func&ț&iei anonime de dup&ă& egal

\end{tcblisting}


Ce ne returnează funcția flip, atunci când îi dăm un singur argument?  O \textit{închidere funcțională} \textbf{peste f}.  Prin urmare, funcția \textbf{mod\_flipped} de mai jos este o închidere funcțională care atunci când primește 2 parametrii va folosi funcția \textbf{mod} (stocată - într-un mod neobservabil -- în \textbf{contextul} său) și va returna restul împărțirii celui de-al doilea la primul.


\begin{tcblisting}{ arc=0pt, outer arc=0pt, listing only, breakable}
mod_flipped = flip mod

\end{tcblisting}


Alte exemple de închideri funcționale:


\begin{tcblisting}{ arc=0pt, outer arc=0pt, listing only, breakable}
mod3 x = mod x 3 -- &î&nchidere func&ț&ional&ă& peste func&ț&ia mod din Prelude &ș&i constanta 3
plus5 x = x + 5  -- &î&nchidere func&ț&ional&ă& peste func&ț&ia (+) &ș&i constanta 5
(+5)             -- aceea&ș&i ca mai sus;
                 --     (+) este o func&ț&ie care prime&ș&te 2 argumente &ș&i se evalueaz&ă& la suma lor
                 --     (+5) este &î&nchiderea func&ț&ional&ă& rezultat&ă& &î&n urma "hardcod&ă&rii" unuia dintre argumente

\end{tcblisting}


Mai multe detalii teoretice despre închideri funcționale, variabile legate/nelegate și combinatori se vor discuta în cadrul cursului.

\begin{tcolorbox}[colback=cyan!5, colframe=cyan!10, breakable]
Puteți găsi și alte exemple de închideri funcționale care se găsesc în acest suport de laborator?
\end{tcolorbox}
\subsection*{ Exerciții updated }
1. Scrieți o funcție ce primește un număr $ x$ și o listă de numere de forma $ [a_1, a_2, ... a_n]$ ca parametrii și calculează $ x - a_1 - a_2 - ... - a_n$.

2. Inversați o listă folosind foldl.

3. Scrieți o închidere funcțională care elimină caracterul 'a' dintr-un string primit ca parametru. Scrieți o implementare ce utilizează o funcție de tip fold și alta care se folosește funcția filter.

4. Implementați o funcție ce primește ca parametru o lista de string-uri și returnează concatenarea lor folosind fold.

5. Se dau două cuvinte de aceeași lungime. Să se numere câte caractere sunt diferite în cele doua șiruri.

6. Implementați o funcție ce primește 2 parametri: un șir de cifre de forma "111222111333112" și un număr K. Această funcție trebuie să facă următorii pași de K ori:
\begin{enumerate}
	\item  se grupează cifrele identice în felul următor "111222111333112" -\textgreater  ["111", "222", "111", "333", "11", "2"]
	\item  pentru fiecare element din lista rezultată se generează un element de forma (numărul de apariții al cifrei din șir, cifra din șir) ex:  ["111", "222", "111", "333", "11", "2"] -\textgreater  [("3", "1"), ("3", "2"), ("3", "1"), ("3", "3"), ("2", "1"), ("1", "2")]
	\item  se face flatten pe lista rezultată la pasul anterior ex: [("3", "1"), ("3", "2"), ("3", "1"), ("3", "3"), ("2", "1"), ("1", "2")] -\textgreater   "313231332112"
\end{enumerate}

\subsection*{ Exerciții}
1. Definiți o închidere funcțională care prefixează [1,2,3] la o listă primită ca parametru.

2. Definiți o funcție de ordin superior care primește o funcție și un număr, și aplică de două ori funcția pe numărul respectiv.

3. Definiți o funcție care primește un operator binar, și întoarce același operator în care ordinea parametrilor a fost inversată \textit{(e.g. 1/3 -\textgreater  3/1)}

4. Implementați și testați funcțiile:
\begin{enumerate}
	\item  foldl
	\item  foldr
	\item  map
	\item  filter
	\item  zipWith
	\item  compunerea de funcții (operatorul \texttt{.} din Haskell)
\end{enumerate}

\begin{tcolorbox}[colback=cyan!5, colframe=cyan!10, breakable]
Dacă nu cunoașteți vreuna dintre funcții, o puteți căuta pe Hoogle pentru a vedea tipul și scopul ei.
\end{tcolorbox}

5. Implementați, folosind foldl sau foldr:
\begin{enumerate}
	\item  map
	\item  filter
\end{enumerate}
\subsection*{ Linkuri utile }
\begin{itemize}
	\item  \texttt{Hoogle - motor de căutare pentru funcții Haskell} \footnote{\url{https://www.haskell.org/hoogle/}}
	\item  \texttt{Higher order functions - haskell.org} \footnote{\url{https://wiki.haskell.org/Higher\_order\_function}}
	\item  \texttt{Higher order functions - learnyouahaskell.com} \footnote{\url{http://learnyouahaskell.com/higher-order-functions}}
\end{itemize}
