\subsection*{ Laborator 08 - Programare in Prolog }

\subsubsection*{ Multimi }
\begin{enumerate}
	\item  Definiti predicatul \texttt{cartesian(L1,L2,R)} care construieste produsul cartezian al \texttt{L1} cu \texttt{L2}
	\item  Definiti predicatul \texttt{union(L1,L2,R)} care construieste reuniunea a doua multimi codificate ca liste.
	\item  Definiti predicatul \texttt{intersection(L1,L2,R)}
	\item  Definiti predicatul \texttt{diff(L1,L2,R)} care construieste diferenta pe multimi intre \texttt{L1} si \texttt{L2}
\end{enumerate}

\subsubsection*{ Permutari, Aranjamente, Combinari }
\begin{enumerate}
	\item  Definiti predicatul \texttt{pow(S,R)} care construieste \texttt{power-set}-ul multimii \texttt{S}.
	\item  Definiti predicatul \texttt{perm(S,R)} care genereaza toate permutarile lui \texttt{S}.
	\item  Definiti predicatul \texttt{ar(K,S,R)} care genereaza toate aranjamentele de dimensiune \texttt{K} cu elemente luate din \texttt{S}
	\item  Definiti predicatul \texttt{comb(K,S,R)} care genereaza toate combinarile de dimensiune \texttt{K} cu elemente luate din \texttt{S}
\end{enumerate}
