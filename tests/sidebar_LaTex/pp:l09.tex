Fie un graf orientat în care fiecare nod are o culoare.
Graful este reprezentat în Prolog printr-o listă ce conține doua elemente [C, E]:
\begin{itemize}
	\item  C: o listă de liste [n, c] cu semnificația că nodul n are culoarea c
	\item  E: o listă de liste [n1, n2] cu semnificația că există o muchie de la nodul n1 la nodul n2 
\end{itemize}
Exemplu:
G = [
[ [1,galben], [2,rosu], [3,albastru], [4,verde], [5,rosu], 
[6,albastru], [7,rosu], [8,galben], [9,albastru], 
[10,mov] ],
[ [1,2],[1,3],[2,6],[3,2],[3,4],[4,5],[5,6],[5,7],[5,8],
[6,4],[6,5],[6,7],[7,8],[8,5],[8,9],[8,10],[9,10] ] ].

\subsubsection*{ Exerciții }

\begin{enumerate}
	\item  Scrieți un predicat getColors care construiește o listă cu toate culorile nodurilor
	\item  Scrieți un predicat getInEdges care construiește o listă cu toate elementele e din E de tipul (\_, X) pentru un X anume
	\item  Scrieți un predicat getOutEdges care construiește o listă cu toate elementele e din E de tipul (X, \_) pentru un X anume
	\item  Scrieți un predicat getUniqueColors (același lucru ca getColors dar fără duplicate)
	\item  Scrieți un predicat getNeighbors ce se folosește de predicatele getInEdges și getOutEdges pentru a construi o listă cu nodurile legate de un nod X
	\item  Scrieți un predicat getPathsOfLength3 care construiește toate drumurile de lungime 3 ce pleacă dintr-un nod X. Valoarea cu care va unifica rezultatul va fi o listă [X, \_, \_]
\end{enumerate}
