\subsection*{ Side-effects }

Programming with side-effects is one defining feature of \textbf{imperative} programming. A procedure produces a \textbf{side-effect} if its call produces a change in the state of a running program (the memory), apart from the returned value.

We (re-)consider the following code discussed in the previous lecture:


\begin{tcblisting}{ arc=0pt, outer arc=0pt, listing only, breakable}
int main (){

	int* v = malloc(sizeof(int)*7);
	v[0] = 1; v[1] = 9; v[2]=4; v[3]= 15; v[4] = 2; v[5] = 6; v[6] = 0;

        show(v);
	insertion_sort(v,7);
        show(v);
}

\end{tcblisting}


The two identical calls on a (hypothetical) \texttt{show} functions will produce different effects. The \textbf{state} of \texttt{v} has changed after the \texttt{insertion\_sort} call.

Programming with side-effects is perhaps one of the most wide-spread styles, and it may seem difficult/unnatural to refrain from using it. However, there are situations where using side-effects may be produce a code prone to bugs.

Consider the following implementation of a \textbf{depth-first traversal} of an oriented graph:


\begin{tcblisting}{ arc=0pt, outer arc=0pt, listing only, breakable}
#include <stdio.h>
#include <stdlib.h>
#include <assert.h> 
 
#define NODE_NO 3
// graph represented as adjacency matrix
typedef struct Graph {
	int** m;
	int nodes;
}* Graph;
 
Graph make_graph (int** m, int n){
	struct Graph* g = (struct Graph*)malloc(sizeof(struct Graph));
	g->m = m;
	g->nodes = n;
	return g;
}
 
int visited[NODE_NO] = {0,0,0};
 
void visit(Graph g, int from){
	visited[from] = 1;
	printf(" Visited %i ",from);
	int i;
	for (i = 0; i<g->nodes; i++){
		if (g->m[from][i] &\&&&\&& !visited[i])
			visit(g,i);
	}
}

int main (){

	int** a = (int**)malloc(NODE_NO*sizeof(int*));
	a[0] = (int*)malloc(NODE_NO*sizeof(int));
	a[0][0] = 0; a[0][1] = 1; a[0][2] = 0;
	a[1] = (int*)malloc(NODE_NO*sizeof(int));
	a[1][0] = 0; a[1][1] = 0; a[1][2] = 1;
	a[2] = (int*)malloc(NODE_NO*sizeof(int));
	a[2][0] = 0; a[2][1] = 0; a[2][2] = 0;

	Graph g = make_graph((int**)a,3);
	visit(g,0);

	//visit(g,1);

}



\end{tcblisting}


\begin{itemize}
	\item  A graph is represented as an \textbf{adjacency matrix}; nodes are identified by integers.
	\item  The code relies on the global array \texttt{visited} which marks those nodes which have been visited during the traversal.
	\item  The procedure \texttt{visit} initiates the traversal: it marks the initial node (\texttt{from}) as visited and recursively visits all adjacent nodes. 
\end{itemize}

\subsubsection*{ Side-effects may easily introduce bugs }

The problem with this implementation is that \texttt{visited} is modified as a side-effect. As it is, two subsequent calls on \texttt{visit} will produce different results:


\begin{tcblisting}{ arc=0pt, outer arc=0pt, listing only, breakable}
int main () {
     visit(g,0);  // produces a correct traversal
     visit(g,1);  // produces an invalid traversal
}

\end{tcblisting}


This happens because \texttt{visited} was not initialised before the second use of \texttt{visit}, and it still holds the visited nodes from the previous traversal.

This bug can be easily solved in different ways:
\begin{itemize}
	\item  make sure \texttt{visited} flags all nodes as not visited before each traversal;
	\item  use a \texttt{visited} array for each traversal;
\end{itemize}

This example is important because:
\begin{itemize}
	\item  side-effects introduce a bug which may be difficult to identify at compile-time (the program will compile) as well as at run-time (the problem only occurs during subsequent traversals, and does not produce a crash).
	\item  the programmer needs to enforce a \textbf{discipline} regarding the scope of a side-effect (e.g. what variables \textbf{outside of \texttt{visit}} are allowed to be modified by \texttt{visit}). Such a discipline may naturally lead to an object-oriented programming style (relying on encapsulation) or to a functional style (limiting or completely forbidding side-effects).
\end{itemize}

\subsubsection*{ Side-effects may be unpredictable }

Consider the following code:

\begin{tcblisting}{ arc=0pt, outer arc=0pt, listing only, breakable}
int fx (){ printf("Hello x !\n"); return 1;}
int fy (){ printf("Hello y !\n"); return 2;}

void test_call(int x, int y){}

int main(){
     test_call(fx(),fy());
}

\end{tcblisting}


Both functions \texttt{fx} and \texttt{fy} produce a side-effect, since they display a message, apart from the returned value. Programmers would generally assume that \texttt{test\_call} will display:

\begin{tcblisting}{ arc=0pt, outer arc=0pt, listing only, breakable}
Hello x !
Hello y !

\end{tcblisting}


However this is not generally the case. In the C standard, there is no specification on the order in which parameters are evaluated. As it happens, on some compilers, \texttt{test\_call} will produce the above output, while on others (e.g. \texttt{ here} \footnote{\url{https://ideone.com/BAsVzC }}), the messages are in a different order.

This is another situation where side-effects may produce hard-to-find bugs.


\subsection*{ Recursion }

Recursion is a natural way of expressing algorithms, for instance the computation of the \textbf{nth Fibbonacci number}:


\begin{tcblisting}{ arc=0pt, outer arc=0pt, listing only, breakable}
int fib (int n){
        if (n<2)
                return n;
        return fib(n-1)+fib(n-2);
}

\end{tcblisting}


\subsubsection*{ Recursion may be inefficient }

The above implementation is inefficient since it triggers an \textbf{exponential} number of \texttt{fib} calls. For instance, the call \texttt{fib(4)} will trigger the following sequence of stack modifications.

In the diagram below, each line represents the state of the \textbf{call stack}. The first function call corresponds to the top of the stack. 


\begin{tcblisting}{ arc=0pt, outer arc=0pt, listing only, breakable}
f 4
f 3 | f 4
f 2 | f 3 | f 4
f 1 | f 2 | f 3 | f 4  //f(1) returns 1;
f 0 | f 2 | f 3 | f 4  //f(0) returns 0;
f 1 | f 3 | f 4        //f(2) returns 1+0;
f 2 | f 4              //f(1) returns 1; f(3) returns 1+1
f 1 | f 2 | f 4             
f 0 | f 2 | f 4        //f(1) returns 1;
[]                     //f(2) returns 1+0(again), f(4) returns 2 + 1

\end{tcblisting}


By solving the recurrence \texttt{T(n) = T(n-1) + T(n-2) + O(1)} we can determine the number of function calls. Conceptually, the problem with this implementation is that it re-computes previously-computed values (e.g. \texttt{f(2)} is computed twice).

\subsubsection*{ Solving the call explosion problem }

We can easily rewrite the implementation of \texttt{fifo} by introducing an accumulator:


\begin{tcblisting}{ arc=0pt, outer arc=0pt, listing only, breakable}
int fib_aux (int f1, int f2, int i, int n){
        if (i == n)
                return f2;
        return fib_aux(f2,f1+f2,i+1,n);
}

int fib2 (int n){
        return fib_aux(0,1,0,n);
}

\end{tcblisting}


The call \texttt{fib2(4)} produces the following sequence of calls:

\begin{tcblisting}{ arc=0pt, outer arc=0pt, listing only, breakable}
fib2 4 
fib_aux 0 1 0 4
fib_aux 1 1 1 4
fib_aux 1 2 2 4
fib_aux 2 3 3 4
fib_aux 3 5 4 4 

\end{tcblisting}


which is linear in the value of \texttt{n}, and does not-recompute values. However, in an imperative language such as C, the call stack can quickly become full for larger \texttt{n}. To solve this problem, many programming languages (e.g. Scala and Haskell, but not C) implement \textbf{tail-call optimisation}. Conceptually, tail-call optimisation relies on the following reasoning.

Consider the following table, in which \texttt{call no} represents the \textit{address} of the function call.

\begin{tcblisting}{ arc=0pt, outer arc=0pt, listing only, breakable}
Call no.   Stack contents            Return address
---------- ------------------------- ------------------------------
1          fib 4                     
2          fib_aux 0 1 0 4           return to call from 1  value 8   
3          fib_aux 1 1 1 4           return to call from 2  value 8
4          fib_aux 1 2 2 4           return to call from 3  value 8
5          fib_aux 2 3 3 4           return to call from 4  value 8
6          fib_aux 3 5 4 4           return to call from 5  value 8  

\end{tcblisting}


For instance, the call \texttt{fib\_aux 1 2 2 4} has address \texttt{5} and after executing it must return to the address \texttt{4}. An intelligent compiler may realise the following reasoning:
\begin{itemize}
	\item  the call \texttt{fib\_aux 1 2 2 4} only returns a value to the subsequent call.
	\item  hence, it makes no sense to keep this call on the stack. Instead, it may \textbf{directly} replace the call \texttt{fib\_aux 1 2 2 4} by \texttt{fib\_aux 3 5 4 4}.
\end{itemize}
 
Following this reasoning, after each recursive call, the stack will contain \textbf{only one call} of \texttt{fib\_aux}, as shown below:

\begin{tcblisting}{ arc=0pt, outer arc=0pt, listing only, breakable}
fib 4 | fib_aux 0 1 0 4              // after first recursive call
fib 4 | fib_aux 1 1 1 4              // after second recursive call
fib 4 | fib_aux 1 2 2 4              // after 3rd recursive call
fib 4 | fib_aux 2 3 3 4              // after 4th recursive call
fib 4 | fib_aux 3 5 4 4              // after 5th recursive call
fib 4 | 5
5 

\end{tcblisting}


Note that the same reasoning \textbf{does not hold for the implementation of \texttt{fib}}, since the recursive calls: \texttt{fib(n-1) + fib(n-2)} require an additional computation (addition) before returning the value. 

For tail-end optimisation to take place: \textbf{a recursive function must be tail-end}. A recursive function is \textbf{tail-end} if it returns a call \textit{to itself} without any additional computation.

Tail-call optimisation in C may be implemented (for \texttt{fib\_aux}) as follows:

\begin{tcblisting}{ arc=0pt, outer arc=0pt, listing only, breakable}
fib_aux (int f1, int f2, int i, int n){
start:
        if (i == n)
                return f2;
        int nf1 = f2;
        int nf2 = f1+f2;
        int ni = i+1;
        int nn = n;       //this instruction may be later eliminated by subsequent optimisation stages.
        f1 = nf1;
        f2 = nf2;
        i = ni;
        nn = n;
        goto start;
}

\end{tcblisting}



\subsection*{ Higher-order functions }

\subsubsection*{ Overview }
Higher-order functions are a programming style which supports \textbf{modularisation}, by replacing recurring programming patterns by a common procedure. Using higher-order functions is not possible in any programming language, and requires \textbf{being able to pass arbitrary functions as parameter}.

Suppose we have a list implementation in C which is \textit{concealed} by the following constructors and observers:


\begin{tcblisting}{ arc=0pt, outer arc=0pt, listing only, breakable}
int head (List l);
List tail (List l);
List cons (int e, List t);
void show (List l);
int size (List l);

\end{tcblisting}


as well as the implementation of the addition and product functions over list elements:

\begin{tcblisting}{ arc=0pt, outer arc=0pt, listing only, breakable}
int sum (List l){
    if (l == Nil)
        return 0;
    return head(l)+sum(tail(l));
}

int product (List l){
    if (l == Nil)
        return 1;
    return head(l)*product(tail(l));
}

\end{tcblisting}


Let us ignore the fact that both recursive functions are not tail end (thus inefficient).

Both implementations follow a common pattern, which only differs in two aspects:
\begin{itemize}
	\item  the \textbf{value} which is returned when \textbf{the list is empty}
	\item  the \textbf{operation} which is performed between the first member of the list, and the recursive call
\end{itemize}

We could express this common pattern by the following implementation:


\begin{tcblisting}{ arc=0pt, outer arc=0pt, listing only, breakable}
int fold (int init, int(*op)(int,int), List l){
    if (isEmpty(l))
        return init;
    return op(head(l),fold(init,op,tail(l)));
}

\end{tcblisting}


The function \texttt{fold} receives as parameter exactly the initial value, and the operation to be performed. Next, addition and product can be rewritten as:


\begin{tcblisting}{ arc=0pt, outer arc=0pt, listing only, breakable}
int op_sum (int x, int y){return x + y;}
int op_product (int x, int y){return x * y;}
int op_size (int x, int y){return 1 + y;}

int sum (List l) {return fold(0,op_sum, l);}
int product (List l) {return fold(1,op_product, l);}

\end{tcblisting}


which is a more modular code. \texttt{fold} is called a \textbf{higher-order function}, because its behaviour is defined with respect to another function (here \textbf{op}). We shall review and improve this definition later on.

One possible criticism to \texttt{fold} is that \textbf{it is not tail-recursive}, however it can be rewritten as:


\begin{tcblisting}{ arc=0pt, outer arc=0pt, listing only, breakable}
int tail_fold (int init, int(*op)(int,int), List l){
    if (l == Nil)
        return init;
    return tail_fold(op(head(l),init),op,tail(l));
}

\end{tcblisting}


However, \texttt{fold} and \texttt{tail\_fold} are not different only with respect to efficiency. They also have different behaviours. Consider the call: 


\begin{tcblisting}{ arc=0pt, outer arc=0pt, listing only, breakable}
fold(i, op, [a,b,c])
op (a, fold(i,op, [b,c]))
op (a, op (b, fold (i, op, [c])))
op (a, op (b, op (c, fold (i, op, []))))
op (a, op (b, op (c, i)))
a op (b op (c op i))

\end{tcblisting}


versus:


\begin{tcblisting}{ arc=0pt, outer arc=0pt, listing only, breakable}
tail_fold(i, op, [a,b,c])
tail_fold(op(a,i),op, [b,c])
tail_fold(op(b,op(a,i)),op, [c])
tail_fold(op(c,op(b,op(a,i))),op,[])
c op (b op (a op i)) 

\end{tcblisting}


The two fold implementations are only the same if 'op' is commutative (addition and product happen to be commutative).

\subsubsection*{ Question }
How can \texttt{sum} be implemented using \texttt{fold} ?

\subsection*{ Programming with higher-order functions is difficult in C }

One criticism to the \texttt{fold}/\texttt{tail\_fold} implementations is that they only work for operations over integers. 
There are many situations where folding may be useful of operations of other types. Consider the following implementation for list concatenation, which cannot be supported by our fold implementations:


\begin{tcblisting}{ arc=0pt, outer arc=0pt, listing only, breakable}
List append (List l1, List l2) {return fold(l2,append_op,l1);}

List append_op(int e, List l2) {return cons(e,l2);}

\end{tcblisting}


For instance, \texttt{append([1,2,3], [4,5,6])} will produce:

\begin{tcblisting}{ arc=0pt, outer arc=0pt, listing only, breakable}
fold([4,5,6], append_op, [1,2,3])
cons(1, fold([4,5,6], append_op, [2,3]))
cons(1, cons(2, fold([4,5,6], append_op, [3])))
cons(1, cons(2, cons(3, fold([4,5,6], append_op, []))))
cons(1, cons(2, cons(3, [4,5,6])))
[1,2,3,4,5,6]

\end{tcblisting}


The above implementation of append cannot rely on our fold, since the type of op supported by \texttt{fold} is: \texttt{int(*op)(int,int)}, while \texttt{append\_op} has type \texttt{List(*op)(int,List)}. There are possible fixes to this issue (e.g. implementing another fold to support such operation types, or using void pointers) however none are clear/natural enough to actually be used in real code.

\subsection*{ Introduction to Haskell }

Haskell is a \textbf{purely-functional} programming language. In short the term \textbf{pure} refers to the fact that \textbf{side-effects are not permitted by design}. Thus, expressions of the form:

\begin{tcblisting}{ arc=0pt, outer arc=0pt, listing only, breakable}
x = expr

\end{tcblisting}

do not produce side-effects. They create a variable \texttt{x} which is \textbf{permanently bound} to the expression \texttt{expr} for the entire program execution. The variable \texttt{x} is \textbf{immutable}: \textbf{it cannot be modified}.

Haskell also \textbf{supports higher-order functions}. The implementation of \texttt{fold} is as follows:


\begin{tcblisting}{ arc=0pt, outer arc=0pt, listing only, breakable}
fold op acc l = if empty(l) then acc else op x (fold op acc xs)

\end{tcblisting}


However, function definitions in Haskell allow specifying \textit{conditional behaviour} depending on how \textbf{objects are constructed} (i.e. base constructors). This is called \textbf{pattern-matching} and deserves a more elaborate discussion (which will be done in a future lecture). Here, we merely state that \texttt{[]} and \texttt{:} (cons) are the base constructors for lists. The code using pattern matching becomes:


\begin{tcblisting}{ arc=0pt, outer arc=0pt, listing only, breakable}
foldr op acc [] = acc
foldr op acc (x:xs) = op x (foldr op acc xs)

\end{tcblisting}


The name (foldr) suggests the order in which the operation \texttt{op} is applied (to the \textbf{r}ight). We also note that \texttt{foldr}, although is not tail-recursive, is actually \textbf{efficient} in Haskell. This is related to the evaluation order, which will be discussed in detail in a future lecture.

In Haskell, \texttt{tail\_fold} is called \texttt{foldl} (to the \textbf{l}eft), and is implemented slightly different:


\begin{tcblisting}{ arc=0pt, outer arc=0pt, listing only, breakable}
foldl op acc [] = acc
foldl op acc (x:xs) = foldl op (op x acc) xs

\end{tcblisting}


The Haskell implementation calls \texttt{(op x acc)} instead of \texttt{(op acc x)} and there is no particular reason for (or against) this choice. Remember that is only holds in Haskell. Other fold functions implemented in other languages may work differently.

The functions \texttt{foldr} and \texttt{foldl} are implemented by-default in Haskell, and can be directly used to define addition, product, or the size of a list:


\begin{tcblisting}{ arc=0pt, outer arc=0pt, listing only, breakable}
sum = foldr (+) 0
prod = foldr (*) 1
size = foldr (1+) 0

\end{tcblisting}


as well as other functions on lists:


\begin{tcblisting}{ arc=0pt, outer arc=0pt, listing only, breakable}
reverse l = foldl (:) [] l
append l1 l2 = foldr (:) l2 l1

\end{tcblisting}

