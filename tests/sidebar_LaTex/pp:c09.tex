\section*{ Formal syntax, Unification, Backtracking, Cut, Negation }

\subsection*{ Prolog syntax }

The basic programming building block in Prolog is the \textbf{clause}. To define it formally, we first introduce:


\begin{tcblisting}{ arc=0pt, outer arc=0pt, listing only, breakable}
<var> ::= alphanumeric sequence starting with capital
<term> ::= <number> | <atom> | <var> | <predicate>(<term1> ..., <term_n>)  

\end{tcblisting}


The definition of \textbf{terms} is slightly restrictive. For instance, terms are also arithmetic expressions (e.g. \texttt{1 + X}), list \textit{patterns} (e.g. [1\textbar[]]), etc. But each such term can be expressed as a \textbf{predicate}, as already illustrated for lists, in the previous lecture.

A clause is defined via the following grammar:


\begin{tcblisting}{ arc=0pt, outer arc=0pt, listing only, breakable}
<clause> ::= <res> :- <goal_1>, ..., <goal_m>.
<res> ::= <predicate>(<var_1>, ... <var_k>)
<goal_i> ::= <predicate>(<var_1>, ... <var_k>) |
             <term> = <term> |                     // unification
             <var> is <term> |                     // arithmetic assignment
             <term> =:= <term> |                   // arithmetic comparison
             true |                                // default satisfying goal
             fail |                                // default-failing goal
             ! |                                   // cut (discussed below)

\end{tcblisting}


Let:

\begin{tcblisting}{ arc=0pt, outer arc=0pt, listing only, breakable}
q(X,Y) :- r(X), p(Y).

\end{tcblisting}


serve as an example. Then \texttt{q(X,Y)} is called \textbf{resolvent}, or \textbf{goal}. Also, in, any query e.g. \texttt{-? q(X,Y).}, \texttt{q(X,Y)} is termed \textbf{goal} (which Prolog tries to \textit{prove}). \texttt{r(X)} and \texttt{p(Y)} are called sub-goals (which Prolog needs to prove, in order to prove \texttt{q(X,Y)}.

There exist special coals, such as \texttt{=} (unification), \texttt{is} (assignment) or \texttt{=:=} (comparison):
\begin{itemize}
	\item  the goal \texttt{\textless term1\textgreater  = \textless term2\textgreater } will be satisfied iff the two terms unify. Under unification \textbf{variables become bound by a substitution}.
	\item  the goal \texttt{\textless var\textgreater  is \textless term\textgreater } will be satisfied iff there is no type error, and will result in binding the variable to the \textbf{evaluation} of the term.
	\item  the goal \texttt{\textless term1\textgreater  =:= \textless term2\textgreater } will be satisfied iff \textbf{the evaluation of each term} yields the same value.
\end{itemize}

We also note that the above syntax is not complete, but covers most of the Prolog programming aspects of this lecture.

\subsubsection*{ Shorthands }

Clauses such as:

\begin{tcblisting}{ arc=0pt, outer arc=0pt, listing only, breakable}
P(X,Y) :- true.

\end{tcblisting}


can also be written as:

\begin{tcblisting}{ arc=0pt, outer arc=0pt, listing only, breakable}
P(X,Y).

\end{tcblisting}


Also, a clause which involves unification such as:

\begin{tcblisting}{ arc=0pt, outer arc=0pt, listing only, breakable}
P(X) :- X = f(a). 

\end{tcblisting}


can also be written as:

\begin{tcblisting}{ arc=0pt, outer arc=0pt, listing only, breakable}
P(f(a),Y).

\end{tcblisting}


Shorthanded unification is often used for lists, and looks similar to Haskell pattern matching, e.g.

\begin{tcblisting}{ arc=0pt, outer arc=0pt, listing only, breakable}
size([],0).
size([_|T],R) :- size(T,R1), R is R1 + 1.

\end{tcblisting}


\subsubsection*{ Common pitfalls }

The following two clauses:


\begin{tcblisting}{ arc=0pt, outer arc=0pt, listing only, breakable}
... :- ..., X=p(A,B) , ...
... :- ..., p(A,B) , ...

\end{tcblisting}


may look similar, but their subgoals are actually very different:
\begin{itemize}
	\item  the first is satisfied if the unification of \texttt{X} with \texttt{p(A,B)} succeeds (hence \texttt{p(A,B)} is merely a term here).
	\item  the second is satisfied if \texttt{p(A,B)} is satisfied.
\end{itemize}

\subsection*{ Unification }

We write $t =_S t'$ to express that \textit{term $t$ unifies with $t'$ under substitution $S$}.

A \textbf{substitution} is a set of pairs $(X,t)$, where $X$ is a variable and $t$ is a term. As a shorthand, we write $t/X$ instead of $(X,t)$.

A \textbf{substitution} $S$ is \textbf{consistent} iff for all pairs $t_1/X$ and $t_2/X$, $t_1 =_S t_2$ (the variable $X$ cannot be bound to different terms which do not unify).


\subsubsection*{ Unification rules }

In what follows, we give some basic unification rules which serve the purpose of illustration. These rules are not implemented per.se. in the Prolog unification process.

$\displaystyle\frac{}{atom =_S atom}$

$\displaystyle\frac{S \cup \{t/X\} \text{ is consistent }}{X =_S t}$

For instance, $X=_{\{cons(H,T)/X\}} cons(1,void)$ since $\{cons(H,T)/X, cons(1,void)/X\}$ is consistent. However $X =_{\{void/X\}} cons(1,void)$ is false ($X$ cannot at the same time be the empty list, and a list with one element).

$\displaystyle\frac{p_1=p_2, n = m, t_1 =_{S_1} t'_1, \ldots, t_n =_{S_1} t'_n, S = S_1 \cup \ldots \cup S_n \text{ is consistent} }{p_1(t_1, ..., t_n) =_S p_2(t'_1, \ldots, t'_m)}$

For instance:

$p(a,q(X,Y),Z) =_S p(X,q(Y,Y),q(X))$, where $S$ is $\{a/X,a/Y,q(a)/Z\}$.

The unification algorithm from Prolog will compute \textbf{the most general substitution} or \textbf{the most general unifier} (MGU). In this lecture, we will not provide a formal definition for the MGU. However, we illustrate the concept via an example.

$p(X,q(X,Z)) =_S p(Y,q(Y,Z)$ for $S = \{Y/X,X/Z\}$ however $S$ is not an MGU. The constraint that $Y$ unifies with $Z$ is superfluous. Here, an MGU is $S = \{Y/X\}$.

\subsubsection*{ Recursive substitutions }

Does \texttt{X=f(X)} produce a valid substitution? Intuitively, such a substitution would contain $f(f(...))/X$. Prolog's unification algorithm is able to detect such recursive substitutions and will not loop. \textbf{Depending on the Prolog implementation (version)} at hand, the unification may fail or succeed.

\subsubsection*{ Goal satisfaction (Backtracking) }

During the satisfaction of a goal, Prolog keeps a current substitution which it iteratively updates. When a sub-goal is satisfied, a new "current substitution" is created.

Let us look in more depth at this.


\begin{tcblisting}{ arc=0pt, outer arc=0pt, listing only, breakable}
contains(E,[E|_]).
contains(E,[H|T]) :- E \= H, contains(E,T).

\end{tcblisting}


The execution tree for the goal \texttt{-? contains(2,[1,2,3]).} is given below:


\begin{tcblisting}{ arc=0pt, outer arc=0pt, listing only, breakable}
contains(2,[1|[2,3]])
    2 \= 1
    contains(2,[2|_]) -> true (continue goal satisfaction)
    contains(2,[2|[3]])   
       2 \= 2 -> false. (re-satisfaction not possible).

\end{tcblisting}


Alternatively, let us look at how \texttt{member} is implemented:


\begin{tcblisting}{ arc=0pt, outer arc=0pt, listing only, breakable}
member(X,[X|_]).
member(X,[_|T]) :- member(X,T).

\end{tcblisting}


The execution tree for \texttt{member(2,[1,2,3])} is shown below:

\begin{tcblisting}{ arc=0pt, outer arc=0pt, listing only, breakable}
member(2,[_|[2,3]])
   member(2,[2|_]) -> true (continue goal satisfaction).
   member(2,[_|[3])
      member(2,[_|[]])
         member(2,[]) -> false (re-satisfaction not possible).

\end{tcblisting}


The difference between \texttt{contains} and \texttt{member} is that, while the former stops once an element in the list has been found, the second continues search. Is there a good motivation for the \texttt{member} implementation?
\begin{itemize}
	\item  build a goal tree for \texttt{member(X,[1,2,3])}
	\item  build a goal tree for \texttt{contains(X,[1,2,3])}.
\end{itemize}


\paragraph{ Backtracking }\hfill\\

Note that goal satisfaction in Prolog is similar to \textit{pruned backtracking}. 

\subsubsection*{ Cut (!) - pruning the goal tree }

Consider the following program:

\begin{tcblisting}{ arc=0pt, outer arc=0pt, listing only, breakable}
f(a).
f(b).
g(a).
g(b).
     
q(X) :- f(X), g(X).

\end{tcblisting}


The proof tree for \texttt{-? q(X)} is:

\begin{tcblisting}{ arc=0pt, outer arc=0pt, listing only, breakable}
f(X)
  X = a (subgoal satisfied). Current substitution is {a/X}
    g(X)
      g(a) true. The goal q(X) is satisfied under {a/X}. Attempt re-satisfaction ...
      ... resatisfaction not possible for g(X) under {a/X}.
  X = b. Current substitution is {b/X}
     g(X)
       g(b) true. Attempt re-satisfaction....
       ... not possible
  not possible to re-satisfy f(X). 
not possible to re-satisfy q(X). stop.  

\end{tcblisting}


In Prolog, the special goal \textit{cut} written \texttt{!}is satisfied according to the following rules:
\begin{enumerate}
	\item  when it is first received, it is \textbf{implicitly satisfied}.
	\item  when there is an attempt to \textbf{re-satisfy it}, it \textbf{fails}.
\end{enumerate}

Let us modify the previous clause to:

\begin{tcblisting}{ arc=0pt, outer arc=0pt, listing only, breakable}
q(X) :- f(X), !, g(X).

\end{tcblisting}


The corresponding proof tree is:

\begin{tcblisting}{ arc=0pt, outer arc=0pt, listing only, breakable}
f(X)
  X = a (subgoal satisfied). Current substitution is {a/X}
    ! (implicitly satisfied)
    g(X)
      g(a) true. The goal q(X) is satisfied under {a/X}. Attempt re-satisfaction ...
      ... resatisfaction not possible for g(X) under {a/X}.
  X = b. Current substitution is {b/X}
    ! (implicitly fails). 
q(X) cannot be re-satisfied. stop

\end{tcblisting}


We turn to a more elaborate example:

\begin{tcblisting}{ arc=0pt, outer arc=0pt, listing only, breakable}
f(a).
f(b).
g(a,c).
g(a,d).
g(b,c).
g(b,d).
  
q(X,Y) :- f(X), !, g(X,Y).
q(test,test).

\end{tcblisting}


The example illustrates a side-effect of the cut semantics: \texttt{q(X,Y)} will not be satisfied for $\{test/X,test/Y\}$. This shows that the effect of cut also extends \textbf{to the current goal}, not only \textbf{the current clause of the current goal}.

Similarly, we have:

\begin{tcblisting}{ arc=0pt, outer arc=0pt, listing only, breakable}
q(test,test) :- !.
q(X,Y) :- f(X), !, g(X,Y).

\end{tcblisting}


where cut prevents the re-satisfaction of \texttt{q} via the second clause (the second cut will not even be reached). 

Also, consider the following code which extends the previous example:

\begin{tcblisting}{ arc=0pt, outer arc=0pt, listing only, breakable}
r(X,Y) :- q(X,Y).
r(1,2).

\end{tcblisting}


The goal \texttt{-? r(X,Y)} will be satisfied for $\{1/X,2/Y\}$, which shows that cut does not have any effect on \texttt{r} - thus, cut should not be interpreted as a \textit{\textbf{global proof tree pruner}}.

\subsubsection*{ Negation }

Prolog implements negation as \textbf{negation-as failure}. This is also called \textbf{the closed-world assumption}. In short, the negation \texttt{not(G)} should be interpreted as \textit{G cannot be proved in the current program}. 

Note that, in First-Order Logic, the negation of a sentence $p$ is interpreted more generally:
\begin{itemize}
	\item  it has its own \textit{proof tree} which is independent of that for $p$
	\item  it is possible to have \textit{theories} (programs) where both $p$ and $~p$ are true. Such theories are called \textbf{inconsistent}. It is also possible that $p$ is provable while $~p$ is not. Such a theory is called \textbf{incomplete}.
\end{itemize}

The Prolog implementation of not is given below:

\begin{tcblisting}{ arc=0pt, outer arc=0pt, listing only, breakable}
not(G) :- G,!,fail.
not(_).

\end{tcblisting}


It is also a good illustration of a \textbf{design pattern} for logical programming, which is often used in Prolog:
\begin{itemize}
	\item  The key property of \texttt{not} is that \textbf{it satisfies without modifying the current substitution}, \textbf{even if G may bound variables}: 

  \begin{itemize}
  	\item  In the first clause, if G is satisfied, cut is reached for the first time (satisfies). Subsequently failure occurs and re-satisfaction is prevented. The second clause is never reached. 
  	\item  However, if G is not satisfied, the cut is never reached, and \texttt{not} succeeds trivially, \textbf{without binding any variable from G}. 
  \end{itemize}
\end{itemize}


The reason while \texttt{X \textbackslash = H} fails in our \texttt{contains} example is that it is actually a syntactic sugar for \texttt{not(X = H)}.

More precisely, note that, in: \texttt{-? H = 1, not(X = H).}
\begin{itemize}
	\item  \texttt{X = H} succeeds hence \texttt{not(X = H)}, fails.
	\item  re-satisfaction of \texttt{not(X = H)} is not possible, hence any supergoal having \texttt{not(X = H)} on a branch will fail also.
\end{itemize}
